%%
%% This is file `sample-acmsmall.tex',
%% generated with the docstrip utility.
%%
%% The original source files were:
%%
%% samples.dtx  (with options: `acmsmall')
%% 
%% IMPORTANT NOTICE:
%% 
%% For the copyright see the source file.
%% 
%% Any modified versions of this file must be renamed
%% with new filenames distinct from sample-acmsmall.tex.
%% 
%% For distribution of the original source see the terms
%% for copying and modification in the file samples.dtx.
%% 
%% This generated file may be distributed as long as the
%% original source files, as listed above, are part of the
%% same distribution. (The sources need not necessarily be
%% in the same archive or directory.)
%%
%% The first command in your LaTeX source must be the \documentclass command.
\documentclass[acmsmall]{acmart}

%%
%% \BibTeX command to typeset BibTeX logo in the docs
\AtBeginDocument{%
  \providecommand\BibTeX{{%
    \normalfont B\kern-0.5em{\scshape i\kern-0.25em b}\kern-0.8em\TeX}}}

%% Rights management information.  This information is sent to you
%% when you complete the rights form.  These commands have SAMPLE
%% values in them; it is your responsibility as an author to replace
%% the commands and values with those provided to you when you
%% complete the rights form.


%%
%% These commands are for a JOURNAL article.

%%
%% Submission ID.
%% Use this when submitting an article to a sponsored event. You'll
%% receive a unique submission ID from the organizers
%% of the event, and this ID should be used as the parameter to this command.
%%\acmSubmissionID{123-A56-BU3}

%%
%% The majority of ACM publications use numbered citations and
%% references.  The command \citestyle{authoryear} switches to the
%% "author year" style.
%%
%% If you are preparing content for an event
%% sponsored by ACM SIGGRAPH, you must use the "author year" style of
%% citations and references.
%% Uncommenting
%% the next command will enable that style.
%%\citestyle{acmauthoryear}

%%
%% end of the preamble, start of the body of the document source.
\begin{document}

%%
%% The "title" command has an optional parameter,
%% allowing the author to define a "short title" to be used in page headers.
\title{Tarea 4}

%%
%% The "author" command and its associated commands are used to define
%% the authors and their affiliations.
%% Of note is the shared affiliation of the first two authors, and the
%% "authornote" and "authornotemark" commands
%% used to denote shared contribution to the research.
\author{Carlos de la Garza Macias}
\email{a01024712@itesm.mx}
\orcid{A01021960}
\affiliation{%
  \institution{Instituto Tecnológico de Estudios Superiores de Monterrey, Campus Santa Fe}
}

%%
%% By default, the full list of authors will be used in the page
%% headers. Often, this list is too long, and will overlap
%% other information printed in the page headers. This command allows
%% the author to define a more concise list
%% of authors' names for this purpose.
%%
%% The abstract is a short summary of the work to be presented in the
%% article.

\begin{abstract}
\section{Abstract}
  El motivo de esta tarea fue para descargar un data set de internet en mi caso fue un data set de Wikipedia y convertirlo a diferentes formatos de grafo. Calcular el tiempo de conversión, la complejidad del algoritmo y después graficarla usando Gephi. 
   
\end{abstract}
%%
%% The code below is generated by the tool at http://dl.acm.org/ccs.cfm.
%% Please copy and paste the code instead of the example below.
%%


%%
%% Keywords. The author(s) should pick words that accurately describe
%% the work being presented. Separate the keywords with commas.
\keywords{grafos,complejidad, algoritmos }


%%
%% This command processes the author and affiliation and title
%% information and builds the first part of the formatted document.
\maketitle

\section{Contenido}
En esta tarea hicimos un programa en C++ en el en el cual bajamos un data set el cual de llamaba wiki-vote de una página de internet y convertimos este data set a diferentes formatos de grafos. Los formatos fueron:

\begin{itemize}
\item GraphML
\item GEXF
\item GDF
\item JSON
\end{itemize}


\subsection{Complejidad}
La complejidad de mis algoritmos para convertir el data set a los diferentes grafos fue lineal lo cual es representado como O(n). Por lo cual los tiempos fueron relativamente rapidos.

\subsection{Tiempo de ejecución}
\begin{itemize}
\item El tiempo de ejecución para conversión a GraphML fue 205.406000 ms 
\item El tiempo de ejecución para conversión a GEXF fue 205.381000 ms 
\item El tiempo de ejecución para conversión a GDF fue 77.415000 ms 
\item El tiempo de ejecución para conversión a GraphSon fue 714.286000 ms 
\end{itemize}
\section{Problemas}
Me enfrente a un par de problemas al enfrentar los algoritmos. Sin embargo, con un poco de ayuda los logre resolver. Por otro lado, una cosa que no pude resolver fue graficar el grafo. Tuve muchas complicaciones usando la aplicación de Gephi.
\section{Conclusiones}
En conclusión, logramos convertir logramos convertir un documento de texto en diferentes formatos de grafos. Espero que en las siguientes clases o en las tareas podamos analizar más profundamente estos sistemas y poder hacer cosas con ellos.

\end{document}
\endinput
%%
%% End of file `sample-acmsmall.tex'.
